% TEMPLATE : FICHE DE REVISION
\documentclass[12pt]{article}

\usepackage[french]{babel}
\usepackage[T1]{fontenc}
\usepackage{graphicx}
\usepackage{framed}
\usepackage{float}
\usepackage[normalem]{ulem}
\usepackage{indentfirst}
\usepackage{amsmath,amsthm,amssymb,amsfonts}
\usepackage[italicdiff]{physics}
\usepackage{wrapfig}
\usepackage{lmodern,mathrsfs}
\usepackage{caption}
\usepackage{subcaption}
\usepackage[dvipsnames]{xcolor}
\usepackage[utf8]{inputenc}
\usepackage[a4paper, top=0.5in,bottom=0.5in, left=0.7in, right=0.7in, footskip=0.3in, includefoot]{geometry}
\usepackage[most]{tcolorbox}
\usepackage{multicol}
\usepackage{tabularx}
\usepackage{setspace}
\usepackage{gensymb}
\usepackage{titlesec}
\usepackage[unicode,psdextra]{hyperref}
\pdfstringdefDisableCommands{\def\varepsilon{\textepsilon}}
\usepackage{bookmark}
\hypersetup{
    colorlinks,
    citecolor=black,
    filecolor=black,
    linkcolor=black,
    urlcolor=black
}


\titleformat*{\section}{\large\bfseries}

\renewcommand{\phi}{\varphi}
\newcommand{\C}{\mathbb{C}}
\newcommand{\K}{\mathbb{K}}
\newcommand{\N}{\mathbb{N}}
\newcommand{\Q}{\mathbb{Q}}
\newcommand{\R}{\mathbb{R}}
\newcommand{\Ne}{\mathbb{N}^{*}}
\newcommand{\Os}[1][]{\underset{#1}{\mathcal{O}}}
\newcommand{\integral}[2]{\ensuremath{\int_{#1}^{#2}}}
\newcommand{\deriv}[3][]{\ensuremath{\frac{d^{#1}#2}{d#3}}}
\let\spn\relax\let\Re\relax\let\Im\relax
\DeclareMathOperator{\Re}{\text{Re}}
\DeclareMathOperator{\Im}{\text{Im}}

\newtheoremstyle{mystyle}{}{}{}{}{\sffamily\bfseries}{.}{ }{}
\newtheoremstyle{cstyle}{}{}{}{}{\sffamily\bfseries}{.}{ }{\thmnote{#3}}
\makeatletter
\renewcommand\qedsymbol{}
\makeatother

\theoremstyle{cstyle}{\newtheorem{definition}{Définition}[section]}
\theoremstyle{cstyle}{\newtheorem{proposition}[definition]{Propriété}}
\theoremstyle{mystyle}{\newtheorem{theorem}[definition]{Théorème}}
\theoremstyle{mystyle}{\newtheorem{lemma}[definition]{Lemme}}
\theoremstyle{mystyle}{\newtheorem{corollary}[definition]{Corollaire}}
\theoremstyle{mystyle}{\newtheorem*{remark}{Remarque}}
\theoremstyle{mystyle}{\newtheorem*{remarks}{Remarques}}
\theoremstyle{mystyle}{\newtheorem*{example}{Exemple}}
\theoremstyle{mystyle}{\newtheorem*{examples}{Exemples}}
\theoremstyle{definition}{\newtheorem*{exercise}{Exercice}}
\theoremstyle{mystyle}{\newtheorem*{methode}{Méthode}}
\theoremstyle{cstyle}{\newtheorem*{cthm}{}}

%Warning environment
\newtheoremstyle{warn}{}{}{}{}{\normalfont}{}{ }{}
\theoremstyle{warn}
\newtheorem*{warning}{\warningsign{0.2}\relax}

%Symbol for the warning environment, designed to be easily scalable
\newcommand{\warningsign}[1]{\tikz[scale=#1,every node/.style={transform shape}]{\draw[-,line width={#1*0.8mm},red,fill=yellow,rounded corners={#1*2.5mm}] (0,0)--(1,{-sqrt(3)})--(-1,{-sqrt(3)})--cycle;
\node at (0,-1) {\fontsize{48}{60}\selectfont\bfseries!};}}

\tcolorboxenvironment{definition}{boxrule=0pt,boxsep=0pt,colback={white},colframe=white,left=8pt,right=8pt,enhanced jigsaw, borderline west={2pt}{0pt}{red},sharp corners,before skip=10pt,after skip=10pt,breakable}
\tcolorboxenvironment{proposition}{boxrule=0pt,boxsep=0pt,colback={white},left=8pt,colframe=white,right=8pt,enhanced jigsaw, borderline west={2pt}{0pt}{Cyan},sharp corners,before skip=10pt,after skip=10pt,breakable}
\tcolorboxenvironment{theorem}{boxrule=0pt,boxsep=0pt,colback={white}, colframe = white, left=8pt,right=8pt,enhanced jigsaw, borderline west={2pt}{0pt}{blue},sharp corners,before skip=10pt,after skip=10pt,breakable}
\tcolorboxenvironment{lemma}{boxrule=0pt,boxsep=0pt,colback={Cyan!10},left=8pt,right=8pt,enhanced jigsaw, borderline west={2pt}{0pt}{Cyan},sharp corners,before skip=10pt,after skip=10pt,breakable}
\tcolorboxenvironment{corollary}{boxrule=0pt,boxsep=0pt,colback={violet!10},left=8pt,right=8pt,enhanced jigsaw, borderline west={2pt}{0pt}{violet},sharp corners,before skip=10pt,after skip=10pt,breakable}
\tcolorboxenvironment{proof}{boxrule=0pt,boxsep=0pt,colback={white},colframe=white,borderline west={2pt}{0pt}{Green!80},enhanced jigsaw,left=8pt,right=8pt,sharp corners,before skip=10pt,after skip=10pt,breakable}
\tcolorboxenvironment{remark}{boxrule=0pt,boxsep=0pt,blanker,borderline west={2pt}{0pt}{CadetBlue!80},left=8pt,right=8pt,before skip=10pt,after skip=10pt,breakable}
\tcolorboxenvironment{remarks}{boxrule=0pt,boxsep=0pt,blanker,borderline west={2pt}{0pt}{Green},left=8pt,right=8pt,before skip=10pt,after skip=10pt,breakable}
\tcolorboxenvironment{example}{boxrule=0pt,boxsep=0pt,blanker,borderline west={2pt}{0pt}{Black},left=8pt,right=8pt,sharp corners,before skip=10pt,after skip=10pt,breakable}
\tcolorboxenvironment{examples}{boxrule=0pt,boxsep=0pt,blanker,borderline west={2pt}{0pt}{Black},left=8pt,right=8pt,sharp corners,before skip=10pt,after skip=10pt,breakable}
\tcolorboxenvironment{cthm}{boxrule=0pt,boxsep=0pt,colback={gray!10},left=8pt,right=8pt,enhanced jigsaw, borderline west={2pt}{0pt}{gray},sharp corners,before skip=10pt,after skip=10pt,breakable}
\tcolorboxenvironment{methode}{boxrule=0pt,boxsep=0pt,colback={Aquamarine!10},left=8pt,right=8pt,enhanced jigsaw, borderline west={2pt}{0pt}{Aquamarine},sharp corners,before skip=10pt,after skip=10pt,breakable}

\setlength{\parindent}{0.2in}
\setlength{\parskip}{0pt}
\setlength{\columnseprule}{0pt}

%\renewcommand{\arraystretch}{1.5}


\title{Chapitre 13 : Systèmes thermodynamiques}
\author{}
\date{}

\setstretch{1.4}

\begin{document}

\maketitle

\begin{minipage}[t]{0.45\textwidth}
	\section{Description d'un système thermodynamique}
	\begin{definition}[Système]
		Portion de l'univers limité par une frontière, où l'on peut distinguer l'intérieur et l'extérieur de cette portion.
	\end{definition}
	\begin{definition}[Système thermodynamique]
		Système possédant un très grand nombre de degrés de liberté.
	\end{definition}
	\begin{definition}[Système fermé]
		Un système est fermé s'il n'échange pas de matière avec l'extérieur. Sinon il est ouvert.
	\end{definition}
	\begin{definition}[État macroscopique / microscopique]
		\underline{L'état microscopique} est la description de la position et de la vitesse de chacun des constituants (impossible à déterminer).
		
		\underline{L'état macroscopique} est donné par des variables thermodynamique qui décrivent l'ensemble.
	\end{definition}

	\begin{definition}[Variable extensive]
		Variable thermodynamique qui décrit le système dans son ensemble et s'ajoute avec l'union de deux systèmes.
		
		Exemples: mole, masse molaire, \dots
	\end{definition}

	
	
\end{minipage}
\hfill
\begin{minipage}[t]{0.45\textwidth}
	\begin{definition}[Mole]
		Une mole correspond à \(\mathcal{N}_A = 6,02 \cdot 10^{23}\) éléments. \(\mathcal{N}_A\) est le nombre d'Avogadro.
	\end{definition}
	\begin{definition}[Masse molaire]
		Masse d'une mole d'un entité (\(M = m \cdot \mathcal{N}_A\)).
	\end{definition}
	\begin{definition}[Variables intensives]
		Variable thermodynamique qui décrit le système localement et ne varie pas lorsque l'on prend l'union de deux systèmes.
		
		Exemples: température, pression, masse volumique, \dots
	\end{definition}

	\begin{definition}[Température]
		Variable intensive qui caractérise l'agitation désordonnée des atomes. 
		
		\underline{Unité :} \textbf{Kelvin (K)}
		
		Ou en \textbf{Celsius} : \(\theta(\degree C) = T(K) - 273,15\)
	\end{definition}

	\begin{definition}[Masse volumique d'un système]
		Quotient de la masse par le volume du système : $\rho = \frac{m}{V}$ en \(kg/m^3\)
	\end{definition}

	
	
\end{minipage}

\newpage
\begin{minipage}[t]{0.45\textwidth}
	
	\begin{definition}[Pression]
		Caractérise la force exercée par un fluide sur une paroi en contact avec lui. Cette force est :
		\[ \vec{F} = P \cdot S \cdot \vec{n} \]
		Avec $P$ la pression du fluide et $S$ la surface en contact.
		
		\underline{Unité :} \begin{itemize}
			\item \textbf{Pascal (\(1Pa = 1N/m^2\))} 	
			\item \textbf{bar \(1bar = 10^5 Pa\)}
		\end{itemize}	
	\end{definition}

	\section{Coefficients thermoélastiques}
	
	\begin{definition}[Coefficient de dilatation isobare]
		Pour un système fermé, le coefficient $\alpha$ est défini par :
		\[
		\alpha = \frac{1}{V} \frac{\partial V}{\partial T}\Big)_{P,n} \text{ en } K^{-1}
		\]
		
		\underline{Interprétation :} lorsque la température augmente de d$T$ à pression constante, le volume augmente de \fbox{d$V = V \alpha$d$T$}
	\end{definition}

	\begin{definition}[Coefficient de compressibilité isotherme]
		Pour un système fermé, le coefficient $\chi$ est défini par :
		\[
		\chi_T = -\frac{1}{V} \frac{\partial V}{\partial P}\Big)_{T,n} \text{ en } {Pa}^{-1}
		\]
		
		\underline{Interprétation :} lorsque la pression augmente de d$P$ à température constante, le volume varie de \fbox{d$V = -V \chi_T$d$P$}
	\end{definition}

	
	

\end{minipage}
\hfill
\begin{minipage}[t]{0.45\textwidth}
	\begin{definition}[Coefficient sans nom $\beta$]
		Pour un système fermé, le coefficient $\beta$ est défini par :
		\[
		\beta = \frac{1}{P} \frac{\partial P}{\partial T}\Big)_{V,n} \text{ en } K^{-1}
		\]
		
		\underline{Interprétation :} lorsque la température augmente de d$T$ à volume constant, alors la pression augmente de \fbox{d$P = P \beta$d$T$}
	\end{definition}

	\section{Modèle de phases simples}
	
	\begin{definition}[Vitesse quadratique d'un gaz]
		Elle est définie par :
		\[
			v^* = \sqrt{\langle \| \vec{v} \|^2 \rangle}
		\]
	\end{definition}

	\begin{proposition}[Énergie cinétique d'un gaz parfait]
		L'énergie cinétique moyenne $\langle e_c \rangle = \frac{1}{2}m{v^*}^2$ d'une molécule de masse $m$ dans un gaz à la température $T$ vaut :
		\[
		\langle e_c \rangle = \frac{3}{2} k_B \cdot T
		\]
		Avec $k_B = 1,38.10^{-23} J.K^{-1}$ constante de Boltzmann.
		
		L'énergie cinétique de $n$ moles de gaz parfait à la température T vaut donc :
		\[
		E_c = \frac{3}{2} n.\mathcal{N}_A.k_B.T
		\]
		On peut également poser $R = k_B. \mathcal{N}_A = 8,314J.K^{-1}.mol^{-1}$ qui est la constante des gaz parfaits.
	\end{proposition}
\end{minipage}

\newpage
\begin{minipage}[t]{0.45\textwidth}
	\begin{proof}[Vitesse quadratique d'un gaz parfait]
		On sait que pour un gaz parfait :
		\[
		\langle e_c \rangle = \frac{1}{2}m{v^*}^2 = \frac{3}{2} k_B T
		\]
		
		Donc \fbox{$v^* = \sqrt{\frac{3k_BT}{m}} = \sqrt{\frac{3RT}{M}}$}
	\end{proof}

	\begin{proposition}[Équation d'état du gaz parfait]
		Un gaz parfait homogène à l'équilibre vérifie :
		\[PV = Nk_BT = nRT\]
		Avec $R = 8,314J.K^{-1}.mol^{-1}$ 
	\end{proposition}

	\begin{definition}[Pression partielle d'un gaz dans un mélange]
		Pression qu'aurait le constituant $i$ d'un mélange s'il était seul dans les mêmes conditions de température et de volume.
	\end{definition}
	\begin{proposition}[Pression d'un mélange idéal de gaz]
		Dans un mélange idéal de gaz, les gaz sont indépendants les un des autres. Donc :
		\[
			P = \sum_k P_k
		\]
	\end{proposition}

	\begin{proposition}[Pression d'un mélange idéal de gaz parfait]
		Pour un mélange de gaz parfait :
		\[
			PV = \sum_k n_kRT = n_{\text{total}}RT
		\]
		Un mélange de gaz parfait se comporte comme un gaz parfait.
	\end{proposition}
\end{minipage}
\hfill
\begin{minipage}[t]{0.45\textwidth}
	\begin{proposition}[Liquide/Solide incompressible/indilatable]
		Conditions :
		\begin{itemize}
			\item Un solide/liquide indilatable à un coefficient $\alpha = 0$
			\item Un solide/liquide incompressible a un coefficient $\chi = 0$
			\item Si un solide/liquide est incompressible et indilatable alors $V = cst$
		\end{itemize}
	\end{proposition}
\end{minipage}

\end{document}
