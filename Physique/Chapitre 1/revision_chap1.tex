% TEMPLATE : FICHE DE REVISION
\documentclass[11pt]{article}

\usepackage[french]{babel}
\usepackage[T1]{fontenc}
\usepackage{graphicx}
\usepackage{framed}
\usepackage{float}
\usepackage[normalem]{ulem}
\usepackage{indentfirst}
\usepackage{amsmath,amsthm,amssymb,amsfonts}
\usepackage[italicdiff]{physics}
\usepackage{wrapfig}
\usepackage{lmodern,mathrsfs}
\usepackage{caption}
\usepackage{subcaption}
\usepackage[dvipsnames]{xcolor}
\usepackage[utf8]{inputenc}
\usepackage[a4paper, top=0.5in,bottom=0.5in, left=0.7in, right=0.7in, footskip=0.3in, includefoot]{geometry}
\usepackage[most]{tcolorbox}
\usepackage{multicol}
\usepackage{tabularx}
\usepackage{setspace}
\usepackage{gensymb}
\usepackage{titlesec}
\usepackage[unicode,psdextra]{hyperref}
\pdfstringdefDisableCommands{\def\varepsilon{\textepsilon}}
\usepackage{pgfplots}
\usepackage{bookmark}
\hypersetup{
	colorlinks,
	citecolor=black,
	filecolor=black,
	linkcolor=black,
	urlcolor=black
}

\graphicspath{ {./image} } 

\titleformat*{\section}{\large\bfseries}
\titleformat*{\subsection}{\normalsize\bfseries}

\renewcommand{\phi}{\varphi}
\newcommand{\C}{\mathbb{C}}
\newcommand{\K}{\mathbb{K}}
\newcommand{\N}{\mathbb{N}}
\newcommand{\Q}{\mathbb{Q}}
\newcommand{\R}{\mathbb{R}}
\newcommand{\Ne}{\mathbb{N}^{*}}
\newcommand{\Os}[1][]{\underset{#1}{\mathcal{O}}}
\newcommand{\integral}[2]{\ensuremath{\int_{#1}^{#2}}}
\newcommand{\deriv}[3][]{\ensuremath{\frac{d^{#1}#2}{d#3}}}
\let\spn\relax\let\Re\relax\let\Im\relax
\DeclareMathOperator{\Re}{\text{Re}}
\DeclareMathOperator{\Im}{\text{Im}}

\newtheoremstyle{mystyle}{}{}{}{}{\sffamily\bfseries}{.}{ }{}
\newtheoremstyle{cstyle}{}{}{}{}{\sffamily\bfseries}{.}{ }{\thmnote{#3}}
\makeatletter
\renewcommand\qedsymbol{}
\makeatother

\theoremstyle{cstyle}{\newtheorem{definition}{Définition}[section]}
\theoremstyle{cstyle}{\newtheorem{proposition}[definition]{Propriété}}
\theoremstyle{cstyle}{\newtheorem{theorem}[definition]{Théorème}}
\theoremstyle{mystyle}{\newtheorem{lemma}[definition]{Lemme}}
\theoremstyle{mystyle}{\newtheorem{corollary}[definition]{Corollaire}}
\theoremstyle{mystyle}{\newtheorem*{remark}{Remarque}}
\theoremstyle{mystyle}{\newtheorem*{remarks}{Remarques}}
\theoremstyle{mystyle}{\newtheorem*{example}{Exemple}}
\theoremstyle{mystyle}{\newtheorem*{examples}{Exemples}}
\theoremstyle{definition}{\newtheorem*{exercise}{Exercice}}
\theoremstyle{mystyle}{\newtheorem*{methode}{Méthode}}
\theoremstyle{cstyle}{\newtheorem*{cthm}{}}

%Warning environment
\newtheoremstyle{warn}{}{}{}{}{\normalfont}{}{ }{}
\theoremstyle{warn}
\newtheorem*{warning}{\warningsign{0.2}\relax}

%Symbol for the warning environment, designed to be easily scalable
\newcommand{\warningsign}[1]{\tikz[scale=#1,every node/.style={transform shape}]{\draw[-,line width={#1*0.8mm},red,fill=yellow,rounded corners={#1*2.5mm}] (0,0)--(1,{-sqrt(3)})--(-1,{-sqrt(3)})--cycle;
		\node at (0,-1) {\fontsize{48}{60}\selectfont\bfseries!};}}

\tcolorboxenvironment{definition}{boxrule=0pt,boxsep=0pt,colback={white},colframe=white,left=8pt,right=8pt,enhanced jigsaw, borderline west={2pt}{0pt}{red},sharp corners,before skip=10pt,after skip=10pt,breakable}
\tcolorboxenvironment{proposition}{boxrule=0pt,boxsep=0pt,colback={white},left=8pt,colframe=white,right=8pt,enhanced jigsaw, borderline west={2pt}{0pt}{Cyan},sharp corners,before skip=10pt,after skip=10pt,breakable}
\tcolorboxenvironment{theorem}{boxrule=0pt,boxsep=0pt,colback={white}, colframe = white, left=8pt,right=8pt,enhanced jigsaw, borderline west={2pt}{0pt}{blue},sharp corners,before skip=10pt,after skip=10pt,breakable}
\tcolorboxenvironment{lemma}{boxrule=0pt,boxsep=0pt,colback={Cyan!10},left=8pt,right=8pt,enhanced jigsaw, borderline west={2pt}{0pt}{Cyan},sharp corners,before skip=10pt,after skip=10pt,breakable}
\tcolorboxenvironment{corollary}{boxrule=0pt,boxsep=0pt,colback={violet!10},left=8pt,right=8pt,enhanced jigsaw, borderline west={2pt}{0pt}{violet},sharp corners,before skip=10pt,after skip=10pt,breakable}
\tcolorboxenvironment{proof}{boxrule=0pt,boxsep=0pt,colback={white},colframe=white,borderline west={2pt}{0pt}{Green!80},enhanced jigsaw,left=8pt,right=8pt,sharp corners,before skip=10pt,after skip=10pt,breakable}
\tcolorboxenvironment{remark}{boxrule=0pt,boxsep=0pt,blanker,borderline west={2pt}{0pt}{CadetBlue!80},left=8pt,right=8pt,before skip=10pt,after skip=10pt,breakable}
\tcolorboxenvironment{remarks}{boxrule=0pt,boxsep=0pt,blanker,borderline west={2pt}{0pt}{Green},left=8pt,right=8pt,before skip=10pt,after skip=10pt,breakable}
\tcolorboxenvironment{example}{boxrule=0pt,boxsep=0pt,blanker,borderline west={2pt}{0pt}{Black},left=8pt,right=8pt,sharp corners,before skip=10pt,after skip=10pt,breakable}
\tcolorboxenvironment{examples}{boxrule=0pt,boxsep=0pt,blanker,borderline west={2pt}{0pt}{Black},left=8pt,right=8pt,sharp corners,before skip=10pt,after skip=10pt,breakable}
\tcolorboxenvironment{cthm}{boxrule=0pt,boxsep=0pt,colback={gray!10},left=8pt,right=8pt,enhanced jigsaw, borderline west={2pt}{0pt}{gray},sharp corners,before skip=10pt,after skip=10pt,breakable}
\tcolorboxenvironment{methode}{boxrule=0pt,boxsep=0pt,colback={Aquamarine!10},left=8pt,right=8pt,enhanced jigsaw, borderline west={2pt}{0pt}{Aquamarine},sharp corners,before skip=10pt,after skip=10pt,breakable}

\setlength{\parindent}{0.2in}
\setlength{\parskip}{0pt}
\setlength{\columnseprule}{0pt}

%\renewcommand{\arraystretch}{1.5}


\title{Chapitre 1 : Bases de l'optique géométrique}
\author{}
\date{}

\setstretch{1.4}

\begin{document}
	
	\maketitle
	
	\begin{minipage}[t]{0.45\textwidth}
		\section{Description de la lumière}
		
		\subsection{Sources de lumières}
		
		\begin{proposition}[Longueur d'onde et fréquence] Elles sont reliées par :
			\fbox{\( f = \frac{c}{\lambda_0}\)}
			
			Avec \(c\) vitesse de la lumière dans le vide.
		\end{proposition}
	
		\begin{definition}[Spectre d'une source lumineuse]
			Ensemble des fréquences contenues dans la lumière émise de cette source.
		\end{definition}
	
		\subsection{Modélisation de la lumière}
		
		\begin{definition}[Optique géométrique]
			La lumière est considérée comme étant constitué de rayons lumineux indépendants.
		\end{definition}
	
		\begin{definition}[Optique ondulatoire]
			La lumière est considérée comme une onde.
		\end{definition}
	
		\begin{definition}[Optique quantique ou photonique]
			La lumière est considérée comme des particules appelées \underline{photons} d'énergie : \fbox{\(E = h.f \)}
			
			Avec \(h = 6,63.10^{-34} J.s\) constante de Planck.
		\end{definition}
		
	\end{minipage}
	\hfill
	\vrule
	\hfill
	\begin{minipage}[t]{0.45\textwidth}
		\section{Propagation des rayons lumineux}
		
		\subsection{Lois générales}
		
		\begin{proposition}[Propagation en ligne droite]
			Dans un milieu transparent homogène et isotrope (les propriétés sont les mêmes quelle que soit la direction), la lumière se propage en ligne droite.
		\end{proposition}
	
		\begin{definition}[Indice optique]
			\[ n(\lambda) = \frac{c}{v} \]
			Avec \(\lambda\) longueur d'onde, \(v\) vitesse de la lumière dans le milieu.
		\end{definition}
	
		\begin{definition}[Miroir]
			Surface qui limite un milieu transparent et qui renvoie la lumière vers ce milieu.
		\end{definition}
	
		\begin{definition}[Dioptre]
			Surface qui sépare deux milieux transparents différents.
		\end{definition}
	
		\begin{theorem}[Lois de Snell-Descartes pour la réflexion]
			Un rayon est réfléchi lorsqu'il arrive sur un miroir ou dioptre et qu'il repart dans le milieu d'où il vient.
			
			Le rayon réfléchi vérifie les propriétés suivantes :
			\begin{itemize}
				\item Il est contenu dans le plan d'incidence
				\item L'angle entre le rayon réfléchi et la normale est égale à celui entre le rayon incident et la normale (\(r = i\))
			\end{itemize}
		\end{theorem}
	\end{minipage}

	

	\newpage
	
	\begin{minipage}[t]{0.45\textwidth}
		\begin{theorem}[Lois de Snell-Descartes pour la réfraction]
			Un rayon est réfracté lorsqu'il arrive sur un dioptre et qu'il le traverse en changeant de direction.
			Le rayon réfracté vérifie les propriétés suivantes: 
			\begin{itemize}
				\item Il est contenu dans le plan d'incidence
				\item L'angle \(r\) entre le rayon réfléchi et la normale est relié à l'angle \(i\) entre le rayon incident et la normale par la loi \fbox{\(n_1 \sin i = n_2 \sin r\)} avec \(n_1, n_2\) indices optiques des deux milieux.
			\end{itemize}
		\end{theorem}
	
		\subsection{Comportement d'un rayon réfracté}
		
		\begin{proposition}[Position par rapport au rayon incident]
			On note \(n_1\) (resp. \(n_2\)) indice optique du milieu de départ (resp. d'arrivée) :
			\begin{itemize}
				\item Si \fbox{\(n_1 > n_2\)}, alors le rayon réfracté \underline{se rapproche} de la normale.
				\item Sinon, le rayon \underline{s'en éloigne}.
			\end{itemize}
		\end{proposition}
	
		\begin{proposition}[Variation de l'angle réfracté en fonction de celui d'incidence]
			Si le rayon arrive en incidence normale, alors le rayon n'est pas dévié. Si \textbf{l'angle d'incidence} \underline{augmente}, alors celui de \textbf{réfraction} aussi.
		\end{proposition}
	
		\begin{proposition}[Réflexion totale]
			Lorsque la lumière passe d'un milieu d'indice fort à un autre d'indice plus faible, il existe un angle d'incidence maximal, appelé \textbf{angle limite de réflexion totale}, qui vaut \(\arcsin \frac{n_2}{n_1}\), au-delà du quel la lumière est totalement réfléchie.
		\end{proposition}
	\end{minipage}
	\hfill
	\vrule
	\hfill
	\begin{minipage}[t]{0.45\textwidth}
		\subsection{Application : fibre optique}
		
		\begin{proposition}[Cône d'acceptance / ouverture numérique]
			Dans une fibre optique, un rayon ne peut se propager que si le rayon incident fait partie du cône d'acceptance, c'est à dire que l'angle incident \(\theta\) est inférieur à une valeur \(\theta_m\). On définit l'ouverture numérique comme le sinus de cet angle.
			
			Pour une fibre optique à saut d'indice, telle que le coeur à un indice \(n_c\) et la gaine \(n_g < n_c\), l'ouverture numérique vaut :
			\[
				ON = \sin \theta_m = \sqrt{n_c^2 - n_g^2}
			\] 	
			
		\end{proposition}			
	
		\begin{proposition}[Dispersion intermodale]
			C'est la différence de temps entre le rayon le plus lent et le rayon le plus rapide qui parcourent la fibre.
			Pour une fibre à saut d'indice, on a :
			\[
				\Delta t = \frac{L.n_c}{c} \frac{n_c - n_g}{n_g}
			\]
			
			Avec \(L\) longueur de la fibre et \(c\) vitesse de la lumière dans le vide.
		\end{proposition}
	\end{minipage}

\end{document}
	