\documentclass[11pt]{article}
\usepackage{amsmath,amsthm,amssymb,amsfonts}
\usepackage[a4paper, top=0.4in,bottom=0.4in, left=0.5in, right=0.5in, footskip=0.3in, includefoot]{geometry}
\usepackage[utf8]{inputenc}
\usepackage[french]{babel}
\usepackage[T1]{fontenc}
\usepackage{setspace}


\title{Colle n°18: exo-types}
\date{}

\renewcommand\thesection{Exercice \arabic{section}.}
\renewcommand{\thesubsection}{\arabic{subsection}.}


\renewcommand{\phi}{\varphi}
\newcommand{\C}{\mathbb{C}}
\newcommand{\K}{\mathbb{K}}
\newcommand{\N}{\mathbb{N}}
\newcommand{\Q}{\mathbb{Q}}
\newcommand{\R}{\mathbb{R}}
\newcommand{\Ne}{\mathbb{N}^{*}}
\newcommand{\Os}[1][]{\underset{#1}{\mathcal{O}}}
\newcommand{\vect}{\text{Vect}}
\newcommand{\integral}[2]{\ensuremath{\int_{#1}^{#2}}}
\newcommand{\deriv}[3][]{\ensuremath{\frac{d^{#1}#2}{d#3}}}
\let\spn\relax\let\Re\relax\let\Im\relax
\DeclareMathOperator{\Re}{\text{Re}}
\DeclareMathOperator{\Im}{\text{Im}}

\setstretch{1.2}

\begin{document}
	
	\maketitle
	
	\section{Commutant d'une matrice (d'après CC-INP n°73)}
	
	{\large\textit{Soit \(C(A)\) l'ensemble des matrices de \(M_2(\R)\) qui commutent avec la matrice \(A = \begin{pmatrix}
				2 & 1\\ 4 & -1
			\end{pmatrix}\)}}
	
	\begin{enumerate}
		\item {\large\textit{Montrer que \(C(A)\) est un sous-espace vectoriel de \(M_2(\R)\).}}\\
		\begin{itemize}
			\renewcommand{\labelitemi}{$\circ$}
			\item \fbox{\(C(A) \subset M_2(\R)\)} car les matrices qui commutent avec \(A\) sont de format \((2,2)\).
			\item \(A \times 0_2 = 0_2 \times A = 0_2\) donc \(0_{2} \in C(A)\)
						
			Donc \fbox{\(C(A) \neq \varnothing\)}
			
			\item Soit \((P, Q) \in C(A)^2, \lambda \in \R\). Montrons que \(\lambda P + Q \in C(A)\) :
			\begin{align*}
			(\lambda P + Q) A &= \lambda PA + QA\\
			&= \lambda A P + AQ \quad (\text{car } (P, Q) \in C(A)^2)\\
			&\text{et}\\
			A (\lambda P + Q) &= \lambda A P + A Q
			\end{align*}
			Donc \(\lambda P + Q\) commute avec \(A\). \fbox{\(\lambda P + Q \in C(A)\)}
		\end{itemize}
		\underline{Donc \(C(A)\) est un sous-espace vectoriel de \(M_2(\R)\)}
		
		\medskip
		
		\item {\large\textit{On note \(V(A)\) le sous-espace vectoriel de \(M_2(\R)\) engendré par les puissance de \(A : V(A) = \text{Vect}(A^k, k \in \N)\). Montrer que \(V(A)\) est un sous-espace vectoriel de \(C(A)\).}}\\
		
		\begin{itemize}
			\renewcommand{\labelitemi}{$\circ$}
			\item \underline{Montrons que \(V(A) \subset C(A)\) :}
			
			 Soit \(B \in V(A)\). \(B\) est une combinaison linéaire finie de \((A^k)_{k\in \N}\). 
			
			Donc il existe \(k \in \N\) tel que \(B = \sum\limits_{i=1}^{k} \lambda_i A^i\), avec \((\lambda_1, \dots, \lambda_k) \in \R^k\).
			
			Alors \(AB = \sum\limits_{i=1}^k \lambda_i A^{i+1}\) et \(BA = \sum\limits_{i=1}^k \lambda_i A^{i+1}\).
			
			Donc \fbox{\(AB = BA\) et \(B \in C(A)\)}
			
			\medskip
			
			\item \underline{Montrons que \(V(A) \neq \varnothing\):}
			
			\(A^0 = I\) et \(0 \times I \in \text{Vect}((A^k)_{k\in \N})\).
			
			Donc \(0_2 \in V(A)\) et \fbox{\(V(A) \neq \varnothing\)}
			
			\item \underline{Montrons que \(V(A)\) est stable par combinaison linéaire :}
			
			Soit \((B, C) \in V(A)^2, \lambda \in \R\) :
			
			Alors :
			\begin{itemize}
				\item \(\exists n \in \N, B = \sum\limits_{i=1}^n \lambda_k A^k, (\lambda_1, \dots, \lambda_n) \in \R^n\)
				\item \(\exists p \in \N, C = \sum\limits_{i=1}^p \mu_k A^k, (\mu_1, \dots, \mu_p) \in \R^p\)
			\end{itemize}
		
			Donc \(\lambda B + C\) est combinaison linéaire de \(A^k\).
			
			Donc \fbox{\(\lambda B + C \in V(A)\)}
		\end{itemize}
		\newpage
	
		\item {\large\textit{Montrer que \(A^2\) est combiansion linéaire de \(A\) et \(I\). Que peut-on en déduire sur \(V(A)\)} ?} 
		
		On a : 
		\[
			A^2 = \begin{pmatrix}
				8 & 1\\
				4 & 5
			\end{pmatrix} = A + 6I
		\]
		
		\underline{Montrons que \(V(A) = \text{Vect}(A, I)\) :}
		
		\begin{itemize}
			\renewcommand{\labelitemi}{$\circ$}
			\item \(\boxed{\supset}\) : Soit \(M \in \text{Vect}(I, A)\).
			Alors il existe \((a,b) \in \R^2\) tel que \(M = aI + bA\).
			
			Donc \(M\) est une combinaison linéaire finie de \((A^k)_{k\in\N}\).
			
			Donc \fbox{\(\vect(I,A) \subset \vect((A^k)_{k\in\N})= V(A)\)}
			
			\medskip
			
			\item \(\boxed{\subset}\) : Soit \(M \in V(A)\).
			
			Donc il existe \(n \in \N, M = \sum\limits_{i=1}^{n} \lambda_k A^k, (\lambda_1, \dots, \lambda_n) \in \R^n\).
			
			On veut montrer \textit{par récurrence}, \(\forall k \in \N, H(k) : ``A^k \text{est combinaison linéaire de} (A,I)"\) :
			\begin{itemize}
				\item \underline{Initialisation : \(k = 0\)} :
				\[
					A^0 = I = 0 A + I 
				\]
				Donc \fbox{\(H(0)\) est vraie.}
				\medskip
				\item \underline{Hérédité : Soit \(k \in \N\) :} On suppose \(H(k)\)vraie, montrons que \(H(k+1)\) est vraie.
				
				\begin{align*}
					A^{k+1} &= A \times A^k\\
					&= A (aI + bA) \quad \text{ par hypothèse de récurrence avec } (a,b) \in \R^2\\
					&= aA + bA^2\\
					&= aA + b(A + 6I)\\
					A^{k+1} &= \boxed{(a+b) A + 6bI}
				\end{align*}
			
				Donc \(H(k+1)\) est vraie.
				
				\medskip
				
				\item \underline{Bilan :} \(\forall k \in \N\), \(A^k\) est combinaison linéaire de \(I\) et \(A\).
				
			\end{itemize}
			\medskip
			
			Or \(\vect(I, A)\) est un espace vectoriel, donc \(\vect((A^k)_{k\in\N}) \subset \vect(I, A)\)
			\end{itemize}
		
			Donc \fbox{\(\vect(I, A) = \vect((A^k)_{k\in\N})\)}
			
			\bigskip
			
			\item {\large\textit{Déterminer une base de \(C(A)\). A-t-on \(C(A) = V(A)\)} ?}
			
			On cherche les matrices \(M\) qui vérifient : \(MA = AM\).
			
			On pose \(M = \begin{pmatrix}
				a & b\\ c & d
			\end{pmatrix}, (a,b,c,d) \in \R^4\)
		
			On obtient :
			\[
				AM = \begin{pmatrix} 2a + c & 2b + d \\ 4a - c & 4b - d\end{pmatrix}
			\]
			\[
				MA = \begin{pmatrix} 2a + 4b & a - b \\ 2c + 4d & c - d \end{pmatrix}
			\]
			
			Il faut donc résoudre le système \(AM = MA\).
			
			On obtient \(C(A) = \left\{ \begin{pmatrix} 3b + d & b \\ 4b & d \end{pmatrix}, (b,d) \in \R^2\right\}\)
			
			Donc \(C(A) = \vect\left(B,I\right), \text{avec } B = \begin{pmatrix} 3 & 1 \\ 4 & 0 \end{pmatrix}\)
			
			\newpage
			
			\underline{Montrons que \(C(A) = V(A)\) :}
			\begin{itemize}
				\renewcommand{\labelitemi}{$\circ$}
				\item \fbox{$\supset$} : Déjà vu dans (2)
				\item \fbox{$\subset$} : 
				\begin{itemize}
					\item \(V(A)\) est un espace vectoriel
					\item \(
						\begin{cases}
							I \in V(A)\\
							B = I + A \in V(A)
						\end{cases}
					\)
					\item Donc \(C(A) \subset V(A)\)
				\end{itemize}
			
				\underline{Bilan :} \(V(A) = Vect(I, B) = C(A)\)
				
			\end{itemize}
			\bigskip
		
			\underline{Montrons que \((I,A)\) est une base de \(C(A)\) :}
			
			\begin{itemize}
				\renewcommand{\labelitemi}{$\circ$}
				\item \fbox{\((I, A)\) engendre \(C(A)\)} (car \(C(A) = \vect(I,A)\))
				\item Montrons que \((I, A)\) est une famille libre : 
				
				Soit \((a,b) \in \R^2\) tel que \(aI + bA = 0_2\), montrons que \(a = b = 0\) :
				\[
					aI + bA = 
					\begin{pmatrix}
						a + 2b & b\\
						4b & a - b
					\end{pmatrix} = 
					\begin{pmatrix}
						0 & 0\\
						0 & 0
					\end{pmatrix}
				\]
				
				On a \(b = 0\) et par substitution on trouve \(a = 0\), donc \fbox{\((I, A)\) est une famille libre}
				
				\end{itemize}
			
				\underline{Donc \((I,A)\) est une base de \(C(A)\)}
	\end{enumerate}

	\section{Application de la décomposition en éléments simples}
	
	\begin{enumerate}
		\item \textit{\large \textbf{Cours :} Soit \(F = \frac{P}{Q}\) sous forme irréductible. Si \(\alpha\) est un pôle simple de \(F\), déterminer sa partie polaire (deux formules et preuves attendues)}
				
		On cherche \(F\) sous la forme \(\displaystyle F = \frac{a}{X - \alpha} + F_1(X)\), avec \(\alpha\) qui n'est pas pôle de \(F_1\)
		
		\bigskip
		
		\underline{Formule 1 :} \(\displaystyle a = \lim\limits_{x \to \alpha} \frac{(X-\alpha)P(X)}{Q(X)}\)
		
		\textit{Preuve :} On cherche \(a\) tel que : 
		\[
			\frac{P(X)}{Q(X)} = \frac{a}{X-\alpha} + F_1(X)
		\]
		\[
			\Longleftrightarrow (X-\alpha)  \frac{P(X)}{Q(X)} = a + (X- \alpha) F_1(X)
		\]
		
		Or \( \lim\limits_{x \to \alpha} a + (x - \alpha)F_1(x) = a\) (car \(\alpha\) n'est pas pôle de \(F_1\)).
		
		Donc par \textit{unicité de la limite} : \fbox{\(\displaystyle\lim\limits_{x \to \alpha}  \frac{(x - \alpha)P(x)}{Q(x)} = a\)}
		
		\bigskip
		
		\underline{Formule 2 :} \(\displaystyle a = \frac{P(\alpha)}{Q'(\alpha)}\)
		
		 \underline{\(Q\) peut s'écrire sous la forme \(Q(X) = (X - \alpha) R(X)\), avec \(R(\alpha) \neq 0\)} (\(\alpha\) est pôle simple de \(F\)).
		 
		 Donc \(Q'(X) = R(X) + (X-\alpha) R'(X)\). Évalué en \(\alpha, \, Q'(\alpha) = R(\alpha) = \frac{Q(\alpha)}{(X-\alpha)}\)
		 
		 Donc \(\displaystyle \frac{P(\alpha)}{Q'(\alpha)} = \frac{(X - \alpha)P(\alpha)}{Q(\alpha)}\)
	 
 	\end{enumerate}
	
\end{document}